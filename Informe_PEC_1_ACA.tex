% Options for packages loaded elsewhere
\PassOptionsToPackage{unicode}{hyperref}
\PassOptionsToPackage{hyphens}{url}
%
\documentclass[
]{article}
\usepackage{lmodern}
\usepackage{amssymb,amsmath}
\usepackage{ifxetex,ifluatex}
\ifnum 0\ifxetex 1\fi\ifluatex 1\fi=0 % if pdftex
  \usepackage[T1]{fontenc}
  \usepackage[utf8]{inputenc}
  \usepackage{textcomp} % provide euro and other symbols
\else % if luatex or xetex
  \usepackage{unicode-math}
  \defaultfontfeatures{Scale=MatchLowercase}
  \defaultfontfeatures[\rmfamily]{Ligatures=TeX,Scale=1}
\fi
% Use upquote if available, for straight quotes in verbatim environments
\IfFileExists{upquote.sty}{\usepackage{upquote}}{}
\IfFileExists{microtype.sty}{% use microtype if available
  \usepackage[]{microtype}
  \UseMicrotypeSet[protrusion]{basicmath} % disable protrusion for tt fonts
}{}
\makeatletter
\@ifundefined{KOMAClassName}{% if non-KOMA class
  \IfFileExists{parskip.sty}{%
    \usepackage{parskip}
  }{% else
    \setlength{\parindent}{0pt}
    \setlength{\parskip}{6pt plus 2pt minus 1pt}}
}{% if KOMA class
  \KOMAoptions{parskip=half}}
\makeatother
\usepackage{xcolor}
\IfFileExists{xurl.sty}{\usepackage{xurl}}{} % add URL line breaks if available
\IfFileExists{bookmark.sty}{\usepackage{bookmark}}{\usepackage{hyperref}}
\hypersetup{
  pdftitle={PEC 1 - Datos Ómicos - ACA},
  pdfauthor={Alberto Castillo Aroca},
  hidelinks,
  pdfcreator={LaTeX via pandoc}}
\urlstyle{same} % disable monospaced font for URLs
\usepackage[margin=1in]{geometry}
\usepackage{color}
\usepackage{fancyvrb}
\newcommand{\VerbBar}{|}
\newcommand{\VERB}{\Verb[commandchars=\\\{\}]}
\DefineVerbatimEnvironment{Highlighting}{Verbatim}{commandchars=\\\{\}}
% Add ',fontsize=\small' for more characters per line
\usepackage{framed}
\definecolor{shadecolor}{RGB}{248,248,248}
\newenvironment{Shaded}{\begin{snugshade}}{\end{snugshade}}
\newcommand{\AlertTok}[1]{\textcolor[rgb]{0.94,0.16,0.16}{#1}}
\newcommand{\AnnotationTok}[1]{\textcolor[rgb]{0.56,0.35,0.01}{\textbf{\textit{#1}}}}
\newcommand{\AttributeTok}[1]{\textcolor[rgb]{0.77,0.63,0.00}{#1}}
\newcommand{\BaseNTok}[1]{\textcolor[rgb]{0.00,0.00,0.81}{#1}}
\newcommand{\BuiltInTok}[1]{#1}
\newcommand{\CharTok}[1]{\textcolor[rgb]{0.31,0.60,0.02}{#1}}
\newcommand{\CommentTok}[1]{\textcolor[rgb]{0.56,0.35,0.01}{\textit{#1}}}
\newcommand{\CommentVarTok}[1]{\textcolor[rgb]{0.56,0.35,0.01}{\textbf{\textit{#1}}}}
\newcommand{\ConstantTok}[1]{\textcolor[rgb]{0.00,0.00,0.00}{#1}}
\newcommand{\ControlFlowTok}[1]{\textcolor[rgb]{0.13,0.29,0.53}{\textbf{#1}}}
\newcommand{\DataTypeTok}[1]{\textcolor[rgb]{0.13,0.29,0.53}{#1}}
\newcommand{\DecValTok}[1]{\textcolor[rgb]{0.00,0.00,0.81}{#1}}
\newcommand{\DocumentationTok}[1]{\textcolor[rgb]{0.56,0.35,0.01}{\textbf{\textit{#1}}}}
\newcommand{\ErrorTok}[1]{\textcolor[rgb]{0.64,0.00,0.00}{\textbf{#1}}}
\newcommand{\ExtensionTok}[1]{#1}
\newcommand{\FloatTok}[1]{\textcolor[rgb]{0.00,0.00,0.81}{#1}}
\newcommand{\FunctionTok}[1]{\textcolor[rgb]{0.00,0.00,0.00}{#1}}
\newcommand{\ImportTok}[1]{#1}
\newcommand{\InformationTok}[1]{\textcolor[rgb]{0.56,0.35,0.01}{\textbf{\textit{#1}}}}
\newcommand{\KeywordTok}[1]{\textcolor[rgb]{0.13,0.29,0.53}{\textbf{#1}}}
\newcommand{\NormalTok}[1]{#1}
\newcommand{\OperatorTok}[1]{\textcolor[rgb]{0.81,0.36,0.00}{\textbf{#1}}}
\newcommand{\OtherTok}[1]{\textcolor[rgb]{0.56,0.35,0.01}{#1}}
\newcommand{\PreprocessorTok}[1]{\textcolor[rgb]{0.56,0.35,0.01}{\textit{#1}}}
\newcommand{\RegionMarkerTok}[1]{#1}
\newcommand{\SpecialCharTok}[1]{\textcolor[rgb]{0.00,0.00,0.00}{#1}}
\newcommand{\SpecialStringTok}[1]{\textcolor[rgb]{0.31,0.60,0.02}{#1}}
\newcommand{\StringTok}[1]{\textcolor[rgb]{0.31,0.60,0.02}{#1}}
\newcommand{\VariableTok}[1]{\textcolor[rgb]{0.00,0.00,0.00}{#1}}
\newcommand{\VerbatimStringTok}[1]{\textcolor[rgb]{0.31,0.60,0.02}{#1}}
\newcommand{\WarningTok}[1]{\textcolor[rgb]{0.56,0.35,0.01}{\textbf{\textit{#1}}}}
\usepackage{graphicx,grffile}
\makeatletter
\def\maxwidth{\ifdim\Gin@nat@width>\linewidth\linewidth\else\Gin@nat@width\fi}
\def\maxheight{\ifdim\Gin@nat@height>\textheight\textheight\else\Gin@nat@height\fi}
\makeatother
% Scale images if necessary, so that they will not overflow the page
% margins by default, and it is still possible to overwrite the defaults
% using explicit options in \includegraphics[width, height, ...]{}
\setkeys{Gin}{width=\maxwidth,height=\maxheight,keepaspectratio}
% Set default figure placement to htbp
\makeatletter
\def\fps@figure{htbp}
\makeatother
\setlength{\emergencystretch}{3em} % prevent overfull lines
\providecommand{\tightlist}{%
  \setlength{\itemsep}{0pt}\setlength{\parskip}{0pt}}
\setcounter{secnumdepth}{-\maxdimen} % remove section numbering

\title{PEC 1 - Datos Ómicos - ACA}
\author{Alberto Castillo Aroca}
\date{1/5/2020}

\begin{document}
\maketitle

{
\setcounter{tocdepth}{2}
\tableofcontents
}
\hypertarget{abstract}{%
\subsection{Abstract}\label{abstract}}

En este estudio se tomó una muestra de 12 ratones machos de las especies
AppNLG-F/NL-G-F y 3xTg-AD-H, en sus versiones transgénicas y salvajes de
entre 7 meses y 1 año de edad, publicadas por Nakabeppu \& Castillo
(2017). Se destaca que las versiones transgénicas cuentan con mutaciones
patogénicas que producen una amiloidosis agresiva y genes humanos
sobreexpresados que generan Alzheimer en los ratones.

Estos datos fueron utilizados para identificar el conjunto de genes
diferencialmente expresados entre las versiones mutantes y de control de
cada una de las especies.

\hypertarget{cuxf3digo-de-los-datos-gse92926}{%
\subsubsection{Código de los datos:
GSE92926}\label{cuxf3digo-de-los-datos-gse92926}}

\hypertarget{repositorio-en-github-httpsgithub.comalcastarotest_omics}{%
\subsubsection{\texorpdfstring{Repositorio en GitHub:
\url{https://github.com/alcastaro/test_omics}}{Repositorio en GitHub: https://github.com/alcastaro/test\_omics}}\label{repositorio-en-github-httpsgithub.comalcastarotest_omics}}

\hypertarget{contexto}{%
\subsection{Contexto}\label{contexto}}

El Alzheimer es la enfermedad neurodegenerativa más frecuente del
planeta. Tan solo en Europa, esta enfermedad tiene una prevalencia del
5.05\%, afectando mayoritariamente a las mujeres (7.13\%). En este
contexto, se prevé un incremento progresivo en la incidencia de esta
enfermedad como consecuencia del envejecimiento poblacional de este
continente.

En este contexto, resulta relevante identificar las condiciones
genéticas que producen esta enfermedad, así como los procesos biológicos
ligados al incremento de la probabilidad de sufrirla.

\hypertarget{objetivos}{%
\subsection{Objetivos}\label{objetivos}}

El objetivo general del estudio de Nakabeppu \& Castillo (2017) fue
clarificar las patologías Abeta, por medio de la comparación de los
genes C4a/C4b, Cd74, Ctss, Gfap, Nfe2l2, Phyhd1, S100b, Tf, Tgfbr2, y
Vim que fueron alterados en las versiones AppNLG-F/NL-G-F y por otra
parte las versiones 3xTg-AD-H contienen genes correspondientes a las
versiones humanas que comunmente están correlacionados con enfermedades
neurodegenerativas.

Cabe destacar que los ratones AppNLG-F/NL-G-F también tienen alteradas
las expresiones los genes Abi3, Apoe, Bin2, Cd33, Ctsc, Dock2, Fcer1g,
Frmd6, Hck, Inpp5D, Ly86, Plcg2, Trem2, Tyrobp.

El objetivo del presente estudio es identificar los genes
diferencialmente expresados con los datos disponibles de las versiones
transgénicas y salvajes de ratones, con el fin de aportar insumos para
el análisis de los procesos biológicos relacionados a la incidencia del
Alzheimer.

\hypertarget{metodologuxeda}{%
\subsection{Metodología}\label{metodologuxeda}}

\hypertarget{naturaleza-de-los-datos}{%
\subsubsection{Naturaleza de los datos}\label{naturaleza-de-los-datos}}

Tanto el estudio original como el presente son comparaciones de grupos
donde el principal factor de análisis es el padecimiento de Alzheimer.
Para esto se utilizaron microarrays de un solo color usando Affymetrix
Mouse Gene 2.0 ST platform.

\hypertarget{muxe9todos-utilizados}{%
\subsubsection{Métodos utilizados}\label{muxe9todos-utilizados}}

Para el presente estudio se utilizó un flujo de trabajo tradicional. En
primer lugar se realizó una exploración de los datos con el fin de
evaluar la calidad de las bases de datos.

Concretamente se identificó la necesidad de normalizar los datos, puesto
que se contó con un nada despreciable ruido de fondo. Esto motivó la
normalización de las bases con el método Robust Multichip Analysis de
Irizarry et al.~(2003)

A continuación se realizó un control de calidad de los datos
normalizados, para proceder a realizar la detección de Batch por medio
de Combat and Principal variation component analysis (PVCA), con lo cual
se pudo determinar que el principal motivo de variabilidad fueron los
residuos y el genotipo.

Posteriormente se realizó una identificación y filtrado de los genes más
variables, ya que estos son los de mayor interés para el estudio. Esto
dio paso a la utilización del paquete limma para seleccionar los genes
diferencialmente expresados.

Finalmente se obtuvieron las anotaciones de los genes identificados y se
realizaron diversas comparaciones múltiples para obtener indicios sobre
la significancia biológica de estos y su funcionalidad en la prevalencia
del Alzheimer.

Cabe destacar que la descripción detallada del input y output de cada
paso se abordan con mayor detalle en el siguiente apartado.

\hypertarget{resultados}{%
\subsection{Resultados}\label{resultados}}

\hypertarget{identificaciuxf3n-de-grupos}{%
\subsubsection{Identificación de
grupos}\label{identificaciuxf3n-de-grupos}}

Para proceder con el estudio se descargaron 12 archivos .cel
provenientes del estudio de Nakabeppu \& Castillo (2017). Estos fueron
categorizados de acuerdo con el grupo y el genotipo al cual pertenecen y
se estableció un nombre corto para estos, como se puede observar en la
siguiente tabla:

\begin{verbatim}
##                                     FileName    Group Genotype AD ShortName
## 1  GSM2440481_APPNL-G-F_1_MoGene-2_0-st_.CEL APPNL.KO    APPNL KO  APPNL.KO
## 2  GSM2440482_APPNL-G-F_2_MoGene-2_0-st_.CEL APPNL.KO    APPNL KO  APPNL.KO
## 3  GSM2440483_APPNL-G-F_3_MoGene-2_0-st_.CEL APPNL.KO    APPNL KO  APPNL.KO
## 4       GSM2440484_APPWT_1_MoGene-2_0-st.CEL APPNL.WT    APPNL WT  APPNL.WT
## 5       GSM2440485_APPWT_2_MoGene-2_0-st.CEL APPNL.WT    APPNL WT  APPNL.WT
## 6       GSM2440486_APPWT_3_MoGene-2_0-st.CEL APPNL.WT    APPNL WT  APPNL.WT
## 7  GSM2440487_3xTg-AD-H_1_MoGene-2_0-st_.CEL  3xTg.KO     3xTg KO   3xTg.KO
## 8  GSM2440488_3xTg-AD-H_2_MoGene-2_0-st_.CEL  3xTg.KO     3xTg KO   3xTg.KO
## 9  GSM2440489_3xTg-AD-H_3_MoGene-2_0-st_.CEL  3xTg.KO     3xTg KO   3xTg.KO
## 10    GSM2440490_non-Tg_1_MoGene-2_0-st_.CEL  3xTg.WT     3xTg WT   3xTg.WT
## 11    GSM2440491_non-Tg_2_MoGene-2_0-st_.CEL  3xTg.WT     3xTg WT   3xTg.WT
## 12    GSM2440492_non-Tg_3_MoGene-2_0-st_.CEL  3xTg.WT     3xTg WT   3xTg.WT
\end{verbatim}

Posteriormente se leyeron los archivos .cel y se realizó un primer
análisis sobre la calidad de los datos con el paquete
arrayQualityMetrics de Bioconductor.

\hypertarget{anuxe1lisis-de-calidad}{%
\subsubsection{Análisis de calidad}\label{anuxe1lisis-de-calidad}}

Los resultados muestran que hay tres bases de datos marcadas con
problemas, específicamente la 3 y la 12 presentan las mayores
deficiencias en cuanto a calidad, por lo que se podría discutir su
exclusión del estudio.

\includegraphics[width=10.26in]{figures/Figura1}

Sin embargo, se procederá a realizar un análisis a mayor profundidad con
una función creada específicamente para este tipo de datos, por medio
del análisis de componentes principales (ACP). Con esta función (ver
anexos) se procedió a calcular la primera y segunda componente y
graficarlas para observar cómo se comportan los datos.

En primer lugar se observa que la primera componente explica el 50\% de
la varianza, sin embargo, se evidencia que ni esta ni la segunda
componente logran discriminar a las dos especies de ratones de una forma
eficiente. Así mismo, no se observa una clara diferenciación de los
individuos con mutaciones y los salvajes.

\includegraphics{Informe_PEC_1_ACA_files/figure-latex/PCA raw data-1.pdf}

Así mismo, en los gráficos de cajas se evidencia que existen diferencias
apreciables en la intensidad de los datos. Esto sugiere la necesidad de
normalizarlos.

\includegraphics{Informe_PEC_1_ACA_files/figure-latex/Boxplot raw data-1.pdf}

\hypertarget{normalizaciuxf3n-y-anuxe1lisis-de-calidad}{%
\subsubsection{Normalización y análisis de
calidad}\label{normalizaciuxf3n-y-anuxe1lisis-de-calidad}}

Para normalizar los datos se utilizó el Robust Multichip Analysis de
Irizarry et al.~(2003), con lo cual se pudo eliminar la variabilidad de
los datos que proceden de fuentes no biológicas como las provenientes de
aspectos técnicos propios del análisis. De este modo también se logra
eliminar ruido de fondo y obtener resultados más claros.

Tras la normalización de los datos, se observa que todas las bases pasan
el análisis de calidad, por lo cual se confirma su idoneidad para el
estudio.

\includegraphics[width=17.51in]{figures/Figura2}

Por otra parte se evidencia que el Análisis de Componentes Principales
ha cambiado, en este las primeras dos componentes explican el 13.4\% y
el 11.8\% respectivamente. Sin embargo, el mayor cambio es que se logra
identificar una clara diferenciación de las especies de ratones y una
ligera discriminación de los que padecen Alzheimer y los sanos.
\includegraphics{Informe_PEC_1_ACA_files/figure-latex/PCA Normalizado-1.pdf}

En este orden de ideas, los gráficos de caja también demuestran que se
ha podido eliminar el ruido de fondo y que los datos son comparables al
tener intensidades equivalentes.

\includegraphics{Informe_PEC_1_ACA_files/figure-latex/Box plot normalizado-1.pdf}

\hypertarget{batch-detection}{%
\subsubsection{Batch Detection}\label{batch-detection}}

Continuando con el análisis, se utilizó el Combat and Principal
variation component analysis (PVCA) para identificar fuentes de
variación de origen técnico, como las diferencias en fechas de
procesamiento e instrumentos.

Los resultados demuestran que la mayor variabilidad procede de los
residuos, seguido del genotipo de los ratones y la interacción Genotipo
y padecimiento de Alzheimer.

\includegraphics{Informe_PEC_1_ACA_files/figure-latex/Estimación de variabilidad-1.pdf}

\hypertarget{identificaciuxf3n-de-genes-con-mayor-variabilidad}{%
\subsubsection{Identificación de genes con mayor
variabilidad}\label{identificaciuxf3n-de-genes-con-mayor-variabilidad}}

Con el fin de optimizar el análisis, se procedió a identificar los genes
con mayor variabilidad y a filtrar los datos. De este modo, se retiraron
los genes que se mantuvieron relativamente constantes y que no estarían
diferencialmente expresados.

\includegraphics{Informe_PEC_1_ACA_files/figure-latex/Niveles de variación estándar-1.pdf}

\hypertarget{filtrado-de-los-genes-con-mayor-variabilidad}{%
\subsubsection{Filtrado de los genes con mayor
variabilidad}\label{filtrado-de-los-genes-con-mayor-variabilidad}}

Los resultados permiten observar que 671 duplicados fueron eliminados,
así como 17973 con baja variabilidad y otros 16710. Al finalizar el
filtrado sólo se mantuvieron 5991 genes.

\begin{verbatim}
## $numDupsRemoved
## [1] 671
## 
## $numLowVar
## [1] 17973
## 
## $numRemoved.ENTREZID
## [1] 16710
\end{verbatim}

\begin{verbatim}
## Features 
##    41345
\end{verbatim}

\begin{verbatim}
## [1] "Filtered genes"
\end{verbatim}

\begin{verbatim}
## [1] 5991
\end{verbatim}

\hypertarget{identificaciuxf3n-de-genes-diferencialmente-expresados}{%
\subsubsection{Identificación de genes diferencialmente
expresados}\label{identificaciuxf3n-de-genes-diferencialmente-expresados}}

Para seleccionar los genes diferencialmente expresados, se procedió a
utilizar el paquete limma de bioconductor. Para esto se creó una matriz
de diseño y otra de comparación. Concretamente la matriz de comparación
establece que se analizará la diferencia en los genes entre las
versiones mutante y salvaje de cada especie (KOvsWT.3xTg y KOvsWT.APP),
y finalmente se realizará una comparación conjunta entre las dos
anteriores, lo cual constituye una interacción de los datos.

\begin{verbatim}
##    A3xTg.KO A3xTg.WT APPNL.KO APPNL.WT
## 1         0        0        1        0
## 2         0        0        1        0
## 3         0        0        1        0
## 4         0        0        0        1
## 5         0        0        0        1
## 6         0        0        0        1
## 7         1        0        0        0
## 8         1        0        0        0
## 9         1        0        0        0
## 10        0        1        0        0
## 11        0        1        0        0
## 12        0        1        0        0
## attr(,"assign")
## [1] 1 1 1 1
## attr(,"contrasts")
## attr(,"contrasts")$Group
## [1] "contr.treatment"
\end{verbatim}

\begin{verbatim}
##           Contrasts
## Levels     KOvsWT.3xTg KOvsWT.APP INT
##   A3xTg.KO           1          0   1
##   A3xTg.WT          -1          0  -1
##   APPNL.KO           0          1  -1
##   APPNL.WT           0         -1   1
\end{verbatim}

Con estas matrices se procedió a estimar un modelo lineal para
identificar los genes diferencialmente expresados y calcularon top
tables de cada comparación, como se puede apreciar en el script adjunto.

De acuerdo con este script se utilizan los genes filtrados, las matrices
de diseño y comparación para obtener estimaciones sobre la diferencia de
expresión de estos genes y obtener p-valores ajustados por medio del
False Discovery Rate. El resultado, son tablas con el Fold-Change,
pruebas t y los p-valores mencionados.

Con estos datos se puede discriminar los genes que realmente están
diferencialmente expresados.

\begin{Shaded}
\begin{Highlighting}[]
\CommentTok{##Modelo lineal}
\KeywordTok{library}\NormalTok{(limma)}
\NormalTok{fit<-}\KeywordTok{lmFit}\NormalTok{(eset_filtered, designMat)}
\NormalTok{fit.main<-}\KeywordTok{contrasts.fit}\NormalTok{(fit, cont.matrix)}
\NormalTok{fit.main<-}\KeywordTok{eBayes}\NormalTok{(fit.main)}


\CommentTok{##Top Tab A3xTg.KO-A3xTg.WT - Efecto Especie 3xTg}
\NormalTok{topTab_KOvsWT}\FloatTok{.3}\NormalTok{xTg <-}\StringTok{ }\KeywordTok{topTable}\NormalTok{ (fit.main, }\DataTypeTok{number=}\KeywordTok{nrow}\NormalTok{(fit.main), }\DataTypeTok{coef=}\StringTok{"KOvsWT.3xTg"}\NormalTok{, }\DataTypeTok{adjust=}\StringTok{"fdr"}\NormalTok{) }


\CommentTok{##Top Tab APPNL.KO-APPNL.WT - Efecto Especie APPNL}
\NormalTok{topTab_KOvsWT.APP <-}\StringTok{ }\KeywordTok{topTable}\NormalTok{ (fit.main, }\DataTypeTok{number=}\KeywordTok{nrow}\NormalTok{(fit.main), }\DataTypeTok{coef=}\StringTok{"KOvsWT.APP"}\NormalTok{, }\DataTypeTok{adjust=}\StringTok{"fdr"}\NormalTok{) }


\CommentTok{##Top Tab (A3xTg.KO-A3xTg.WT) - (APPNL.KO-APPNL.WT) - Efecto interacción}
\NormalTok{topTab_INT  <-}\StringTok{ }\KeywordTok{topTable}\NormalTok{ (fit.main, }\DataTypeTok{number=}\KeywordTok{nrow}\NormalTok{(fit.main), }\DataTypeTok{coef=}\StringTok{"INT"}\NormalTok{, }\DataTypeTok{adjust=}\StringTok{"fdr"}\NormalTok{) }
\end{Highlighting}
\end{Shaded}

\begin{Shaded}
\begin{Highlighting}[]
\KeywordTok{print}\NormalTok{(}\KeywordTok{head}\NormalTok{(topTab_INT))}
\end{Highlighting}
\end{Shaded}

\begin{verbatim}
##              logFC  AveExpr          t      P.Value    adj.P.Val         B
## 17266967 -3.151477 5.942125 -14.027454 6.796544e-10 3.934912e-06 11.754695
## 17471541 -3.151814 5.550521 -13.366541 1.313608e-09 3.934912e-06 11.275355
## 17233226 -2.795399 5.315233  -8.777863 3.230895e-07 3.934743e-04  6.812324
## 17343918 -1.896725 7.036292  -8.721349 3.501156e-07 3.934743e-04  6.742092
## 17483615 -2.061306 5.524614  -8.640570 3.929665e-07 3.934743e-04  6.640940
## 17280184 -2.631225 7.257245  -8.638623 3.940654e-07 3.934743e-04  6.638490
\end{verbatim}

Posteriormente se procedió a obtener la anotación de los resultados,
para identificar el nombre real de los genes seleccionados y de este
modo tener información más amplia de cada probeid.

\hypertarget{ejemplo-de-los-datos}{%
\paragraph{Ejemplo de los datos}\label{ejemplo-de-los-datos}}

\begin{verbatim}
##    PROBEID SYMBOL ENTREZID                                 GENENAME
## 1 17211286   Ly96    17087                    lymphocyte antigen 96
## 2 17211347 Tfap2b    21419           transcription factor AP-2 beta
## 3 17211382  Efhc1    71877 EF-hand domain (C-terminal) containing 1
## 4 17211416 Khdc1a   368204                  KH domain containing 1A
## 5 17211429 Khdc1b    98582                  KH domain containing 1B
\end{verbatim}

A continuación se realizó un gráfico de volcan para observar la
expresión diferencial de estos genes. Cabe destacar que este tipo de
gráfico realiza una comparación del logaritmo del Fold Change y el
logaritmo negativo del p-valor ajustado. De este modo se puede observar
la proporción de genes diferencialmente expresados y con un alto Fold
Change.

\begin{verbatim}
## 'select()' returned 1:1 mapping between keys and columns
\end{verbatim}

\includegraphics{Informe_PEC_1_ACA_files/figure-latex/Volcano Plot-1.pdf}

El gráfico demuestra que en efecto existen genes diferencialmente
expresados, destacándose G530011O, Zfp125903025P, entre otros. No
obstante, con el fin de identificar cuáles fueron seleccionados de cada
especie, se realizó una comparación múltiple que permitió contabilizar
la cantidad de genes subexpresados y sobreexpresados. Para esto fue
utilizado el modelo lineal del apartado anterior, distinguiendo los
genes con un p-valor ajustado inferior a 0.1.

Los datos demuestran que para los ratones 3xTg hubo 14 genes
subexpresados y 5 sobreexpresados, mientras que para los APP hubo 1
subexpresado y 29 sobreexpresados. Por otra parte, la interacción entre
ambos resultados permitió distinguir 36 genes subexpresados y 6
sobreexpresados.

Se debe recordar que los ratones de la especie 3xtg poseen genes
equiparables a los de los humanos con enfermedades neurodegenerativas.
En consecuencia, se puede inferir que una baja expresividad de estos
genes estarían generando Alzheimer. Por otra parte para los ratones
AppNLG-F/NL-G-F se evidencia que la mayoría de los genes están
sobreexpresados, lo cual concuerda con el estudio de Nakabeppu \&
Castillo (2017), según el cual una amiloidosis agresiva se encuentra
relacionada con el padecimiento de Alzheimer y otras enfermedades
neuronales.

\begin{verbatim}
##        KOvsWT.3xTg KOvsWT.APP  INT
## Down            14          1   36
## NotSig        5972       5961 5949
## Up               5         29    6
\end{verbatim}

Resulta interesante observar el diagrama de Venn de los resultados
anteriores. Según este, se observa que las dos versiones de ratones no
comparten genes diferencialmente expresados directamente, lo cual se
puede explicar por sus diferencias de genotipo. Sin embargo, se
evidencia que en la interacción es donde se presenta la mayor cantidad
de genes con 42.

\includegraphics{Informe_PEC_1_ACA_files/figure-latex/Diagrama Venn-1.pdf}

Adicionalmente, para observar la relación entre estos genes, se procedió
a construir un heatmap jerarquizado. Los resultados demuestran que a
grandes razgos se pueden diferenciar 4 grupos de genes, con similitudes
en su comportamiento y anotación.

No obstante, cabe destacar que la interpretación biológica de este
resultado queda por fuera del presente informe técnico.

\begin{verbatim}
## 'select()' returned 1:1 mapping between keys and columns
\end{verbatim}

\begin{verbatim}
## 
## Attaching package: 'gplots'
\end{verbatim}

\begin{verbatim}
## The following object is masked from 'package:IRanges':
## 
##     space
\end{verbatim}

\begin{verbatim}
## The following object is masked from 'package:S4Vectors':
## 
##     space
\end{verbatim}

\begin{verbatim}
## The following object is masked from 'package:stats':
## 
##     lowess
\end{verbatim}

\includegraphics{Informe_PEC_1_ACA_files/figure-latex/Heatmap-1.pdf}

\hypertarget{significancia-bioluxf3gica}{%
\subsubsection{Significancia
biológica}\label{significancia-bioluxf3gica}}

Tras los análisis anteriores, se avanzó un ejercicio de caraterización
de los procesos biológicos y las funciones que generalmente desempeñan
los genes en cuestión. No obstante, para manejar una visión más amplia,
se seleccionaron todos los genes con u p-valor ajustado inferior a 0.15,
lo cual permitió contar con más datos para identificar estos procesos.

En este análisis se utilizaron las 3 top tables generadas anteriormente
y se realizó una anotación del EntrezID.

\begin{verbatim}
## 'select()' returned 1:1 mapping between keys and columns
## 'select()' returned 1:1 mapping between keys and columns
## 'select()' returned 1:1 mapping between keys and columns
\end{verbatim}

\begin{verbatim}
## KOvsWT.3xTg  KOvsWT.APP         INT 
##          80         127          71
\end{verbatim}

Como se puede observar, el resultado se elevó a 80 genes de los ratones
3xTg, 127 de los APP y 71 en la interacción. En cambio, a continuación
se presentan los resultados del enrichment pathway con los cuales están
asociados los genes.

En primer lugar se observa que las versiones de ratón 3xTg tienen 15
pathways, entre los que se destacan: Map kinases, interleukin-17
signaling, map kinase activation, entre otras.

En segundo lugar, se evidencia que los ratones APP, tienen menos
pathways, apenas 3, entre los que se destacan: traffiking and processing
of endorsal TLR, toll like recpetor cascades y ROS and RNS production in
phagocytes.

Cabe destacar que los gráficos de red y de cajas ilustran las
comparaciones entre las versiones mutantes y normales de los ratones y
que por lo tanto estos procesos biológicos estarían relacionados a estas
diferencias y por ende al padecimiento de Alzheimer, sin embargo, el
análisis a profundidad de sus implicaciones debe ser llevado a cabo por
profesionales de Genética y Biología.

\begin{verbatim}
## 
\end{verbatim}

\begin{verbatim}
## ReactomePA v1.32.0  For help: https://guangchuangyu.github.io/ReactomePA
## 
## If you use ReactomePA in published research, please cite:
## Guangchuang Yu, Qing-Yu He. ReactomePA: an R/Bioconductor package for reactome pathway analysis and visualization. Molecular BioSystems 2016, 12(2):477-479
\end{verbatim}

\begin{verbatim}
## ##########################
## Comparison:  KOvsWT.3xTg 
##                          ID
## R-MMU-450282   R-MMU-450282
## R-MMU-448424   R-MMU-448424
## R-MMU-450294   R-MMU-450294
## R-MMU-202670   R-MMU-202670
## R-MMU-168164   R-MMU-168164
## R-MMU-9018677 R-MMU-9018677
##                                                        Description GeneRatio
## R-MMU-450282  MAPK targets/ Nuclear events mediated by MAP kinases      3/25
## R-MMU-448424                              Interleukin-17 signaling      3/25
## R-MMU-450294                                 MAP kinase activation      3/25
## R-MMU-202670                                  ERKs are inactivated      2/25
## R-MMU-168164                   Toll Like Receptor 3 (TLR3) Cascade      3/25
## R-MMU-9018677                     Biosynthesis of DHA-derived SPMs      2/25
##               BgRatio       pvalue    p.adjust      qvalue          geneID
## R-MMU-450282  28/8772 6.391982e-05 0.007031180 0.004911733 Fos/Dusp6/Dusp4
## R-MMU-448424  55/8772 4.865400e-04 0.008117422 0.005670543 Fos/Dusp6/Dusp4
## R-MMU-450294  55/8772 4.865400e-04 0.008117422 0.005670543 Fos/Dusp6/Dusp4
## R-MMU-202670  13/8772 5.966834e-04 0.008117422 0.005670543     Dusp6/Dusp4
## R-MMU-168164  68/8772 9.069920e-04 0.008117422 0.005670543 Fos/Dusp6/Dusp4
## R-MMU-9018677 16/8772 9.131757e-04 0.008117422 0.005670543      Hpgd/Ptgs2
##               Count
## R-MMU-450282      3
## R-MMU-448424      3
## R-MMU-450294      3
## R-MMU-202670      2
## R-MMU-168164      3
## R-MMU-9018677     2
\end{verbatim}

\includegraphics{Informe_PEC_1_ACA_files/figure-latex/resultado significancia-1.pdf}
\includegraphics{Informe_PEC_1_ACA_files/figure-latex/resultado significancia-2.pdf}

\begin{verbatim}
## ##########################
## Comparison:  KOvsWT.APP 
##                          ID                                 Description
## R-MMU-1679131 R-MMU-1679131 Trafficking and processing of endosomal TLR
## R-MMU-168898   R-MMU-168898                 Toll-like Receptor Cascades
## R-MMU-1222556 R-MMU-1222556        ROS and RNS production in phagocytes
##               GeneRatio  BgRatio       pvalue   p.adjust     qvalue
## R-MMU-1679131      3/78  12/8772 0.0001404609 0.01561519 0.01402508
## R-MMU-168898       7/78 133/8772 0.0001697303 0.01561519 0.01402508
## R-MMU-1222556      4/78  36/8772 0.0002745664 0.01684007 0.01512524
##                                               geneID Count
## R-MMU-1679131                      Ctss/Tlr7/Unc93b1     3
## R-MMU-168898  Ly86/Ctss/Cd180/Itgb2/Tlr7/Fos/Unc93b1     7
## R-MMU-1222556             Atp6v0d2/Cybb/Slc11a1/Ncf2     4
\end{verbatim}

\includegraphics{Informe_PEC_1_ACA_files/figure-latex/resultado significancia-3.pdf}
\includegraphics{Informe_PEC_1_ACA_files/figure-latex/resultado significancia-4.pdf}

\hypertarget{limitaciones-del-estudio}{%
\subsection{Limitaciones del estudio}\label{limitaciones-del-estudio}}

Se destaca que en el presente estudio se hizo uso de datos existentes,
compartidos en GEO. Este hecho limita las posibilidades de la
investigación, en el sentido de que el experimiento ya fue diseñado y
por lo tanto, el análisis se ve forzado a trabajar con la información
disponible.

No obstante, se evidencia que los datos presentaron una buena calidad.

\hypertarget{conclusiones}{%
\subsection{Conclusiones}\label{conclusiones}}

Se debe recordar que los ratones de la especie 3xtg poseen genes
equiparables a los de los humanos con enfermedades neurodegenerativas.
En consecuencia, se puede inferir que una baja expresividad de estos
genes estarían generando Alzheimer. Por otra parte, para los ratos
AppNLG-F/NL-G-F se evidencia que la mayoría de los genes están
sobreexpresados, lo cual concuerda con el estudio de Nakabeppu \&
Castillo (2017), según el cual una amiloidosis agresiva se encuentra
relacionada con el padecimiento de Alzheimer y otras enfermedades
neuronales.

Por otra parte, se destaca que conclusiones de mayor profundidad serían
aportadas por un equipo de biólogos y genetistas que trabajarían con
base en el análisis bioinformático realizado.

\hypertarget{referencias}{%
\subsection{Referencias}\label{referencias}}

\begin{itemize}
\tightlist
\item
  Castillo, E., Leon, J., Mazzei, G. et al.~Comparative profiling of
  cortical gene expression in Alzheimer's disease patients and mouse
  models demonstrates a link between amyloidosis and neuroinflammation.
  Sci Rep 7, 17762 (2017).
  \url{https://doi.org/10.1038/s41598-017-17999-3}
\end{itemize}

\hypertarget{anexos}{%
\subsection{Anexos}\label{anexos}}

\hypertarget{cuxf3digo-en-r}{%
\subsubsection{Código en R}\label{cuxf3digo-en-r}}

\begin{Shaded}
\begin{Highlighting}[]
\CommentTok{### -- Definiendo el directorio }\AlertTok{###}
\CommentTok{#Estudio seleccionado}
\CommentTok{#GSE92926}

\CommentTok{#Comparative profiling of cortical gene expression in Alzheimer's }
\CommentTok{#disease patients and mouse models demonstrates a link between amyloidosis }
\CommentTok{#and neuroinflammation}

\KeywordTok{setwd}\NormalTok{(}\StringTok{"C:/Users/Alberto Castillo/Google Drive/Academico/Máster/2019-3/Análisis de datos ómicos/PEC 1 - Omics"}\NormalTok{)}

\CommentTok{##Estableciendo Knitr}
\KeywordTok{library}\NormalTok{(knitr)}
\NormalTok{knitr}\OperatorTok{::}\NormalTok{opts_chunk}\OperatorTok{$}\KeywordTok{set}\NormalTok{(}\DataTypeTok{echo =} \OtherTok{TRUE}\NormalTok{, }\DataTypeTok{message =} \OtherTok{FALSE}\NormalTok{, }\DataTypeTok{warning =} \OtherTok{FALSE}\NormalTok{, }
                      \DataTypeTok{comment =} \OtherTok{NA}\NormalTok{, }\DataTypeTok{prompt =} \OtherTok{TRUE}\NormalTok{, }\DataTypeTok{tidy =} \OtherTok{FALSE}\NormalTok{, }
                      \DataTypeTok{fig.width =} \DecValTok{7}\NormalTok{, }\DataTypeTok{fig.height =} \DecValTok{7}\NormalTok{, }\DataTypeTok{fig_caption =} \OtherTok{TRUE}\NormalTok{,}
                      \DataTypeTok{cache=}\OtherTok{FALSE}\NormalTok{)}
\KeywordTok{Sys.setlocale}\NormalTok{(}\StringTok{"LC_TIME"}\NormalTok{, }\StringTok{"C"}\NormalTok{)}


\CommentTok{##Establecinedo printr}
\ControlFlowTok{if}\NormalTok{(}\OperatorTok{!}\NormalTok{(}\KeywordTok{require}\NormalTok{(printr))) \{}
  \KeywordTok{install.packages}\NormalTok{(}
    \StringTok{'printr'}\NormalTok{,}
    \DataTypeTok{type =} \StringTok{'source'}\NormalTok{,}
    \DataTypeTok{repos =} \KeywordTok{c}\NormalTok{(}\StringTok{'http://yihui.name/xran'}\NormalTok{, }\StringTok{'http://cran.rstudio.com'}\NormalTok{)}
\NormalTok{  )}
\NormalTok{\}}

\CommentTok{###Definiendo los grupos###}
\NormalTok{set.target=}\KeywordTok{as.data.frame}\NormalTok{(}\KeywordTok{matrix}\NormalTok{(}\DecValTok{0}\NormalTok{,}\DataTypeTok{nrow=}\DecValTok{12}\NormalTok{,}\DataTypeTok{ncol=}\DecValTok{5}\NormalTok{))}
\KeywordTok{colnames}\NormalTok{(set.target)=}\KeywordTok{c}\NormalTok{(}\StringTok{"FileName"}\NormalTok{,}\StringTok{"Group"}\NormalTok{,}\StringTok{"Genotype"}\NormalTok{,}\StringTok{"AD"}\NormalTok{,}\StringTok{"ShortName"}\NormalTok{)}
\NormalTok{set.target}

\NormalTok{set.target[,}\DecValTok{1}\NormalTok{]=}\KeywordTok{list.files}\NormalTok{(}\StringTok{"./data"}\NormalTok{)[}\OperatorTok{-}\DecValTok{1}\NormalTok{]}
\NormalTok{set.target[,}\DecValTok{3}\NormalTok{]=}\KeywordTok{c}\NormalTok{(}\KeywordTok{rep}\NormalTok{(}\StringTok{"APPNL"}\NormalTok{,}\DecValTok{6}\NormalTok{),}\KeywordTok{rep}\NormalTok{(}\StringTok{"3xTg"}\NormalTok{,}\DecValTok{6}\NormalTok{))}
\NormalTok{set.target[,}\DecValTok{4}\NormalTok{]=}\KeywordTok{rep}\NormalTok{(}\KeywordTok{c}\NormalTok{(}\KeywordTok{rep}\NormalTok{(}\StringTok{"KO"}\NormalTok{,}\DecValTok{3}\NormalTok{),}\KeywordTok{rep}\NormalTok{(}\StringTok{"WT"}\NormalTok{,}\DecValTok{3}\NormalTok{)),}\DecValTok{2}\NormalTok{)}
\NormalTok{set.target[,}\DecValTok{2}\NormalTok{]=}\KeywordTok{paste}\NormalTok{(set.target[,}\DecValTok{3}\NormalTok{],set.target[,}\DecValTok{4}\NormalTok{],}\DataTypeTok{sep =} \StringTok{"."}\NormalTok{)}
\NormalTok{set.target[,}\DecValTok{5}\NormalTok{]=set.target[,}\DecValTok{2}\NormalTok{]}

\NormalTok{targets=set.target}

\KeywordTok{write.csv2}\NormalTok{(}\DataTypeTok{x =}\NormalTok{ set.target,}\DataTypeTok{file=}\StringTok{"./data/targets.csv"}\NormalTok{)}

\CommentTok{##Leyendo archivos CEL}
\KeywordTok{library}\NormalTok{(oligo)}
\NormalTok{celFiles <-}\StringTok{ }\KeywordTok{list.celfiles}\NormalTok{(}\StringTok{"./data"}\NormalTok{, }\DataTypeTok{full.names =} \OtherTok{TRUE}\NormalTok{)}
\KeywordTok{library}\NormalTok{(Biobase)}
\NormalTok{my.targets <-}\KeywordTok{read.AnnotatedDataFrame}\NormalTok{(}\KeywordTok{file.path}\NormalTok{(}\StringTok{"./data"}\NormalTok{,}\StringTok{"targets.csv"}\NormalTok{), }
                                     \DataTypeTok{header =} \OtherTok{TRUE}\NormalTok{, }\DataTypeTok{row.names =} \DecValTok{1}\NormalTok{, }
                                     \DataTypeTok{sep=}\StringTok{";"}\NormalTok{) }
\NormalTok{rawData <-}\StringTok{ }\KeywordTok{read.celfiles}\NormalTok{(celFiles, }\DataTypeTok{phenoData =}\NormalTok{ my.targets)}


\CommentTok{##Cambiando los nombres}
\NormalTok{my.targets}\OperatorTok{@}\NormalTok{data}\OperatorTok{$}\NormalTok{ShortName->}\KeywordTok{rownames}\NormalTok{(}\KeywordTok{pData}\NormalTok{(rawData))}
\KeywordTok{colnames}\NormalTok{(rawData) <-}\KeywordTok{rownames}\NormalTok{(}\KeywordTok{pData}\NormalTok{(rawData)) }

\KeywordTok{head}\NormalTok{(rawData)}


\CommentTok{###Análisis de calidad de los datos###}
\KeywordTok{library}\NormalTok{(arrayQualityMetrics)}
\KeywordTok{arrayQualityMetrics}\NormalTok{(rawData)}


\CommentTok{##Análisis de Componentes principales}
\KeywordTok{library}\NormalTok{(ggplot2)}
\KeywordTok{library}\NormalTok{(ggrepel)}
\NormalTok{plotPCA3 <-}\StringTok{ }\ControlFlowTok{function}\NormalTok{ (datos, labels, factor, title, scale,colores, }\DataTypeTok{size =} \FloatTok{1.5}\NormalTok{, }\DataTypeTok{glineas =} \FloatTok{0.25}\NormalTok{) \{}
\NormalTok{  data <-}\StringTok{ }\KeywordTok{prcomp}\NormalTok{(}\KeywordTok{t}\NormalTok{(datos),}\DataTypeTok{scale=}\NormalTok{scale)}
  \CommentTok{# plot adjustments}
\NormalTok{  dataDf <-}\StringTok{ }\KeywordTok{data.frame}\NormalTok{(data}\OperatorTok{$}\NormalTok{x)}
\NormalTok{  Group <-}\StringTok{ }\NormalTok{factor}
\NormalTok{  loads <-}\StringTok{ }\KeywordTok{round}\NormalTok{(data}\OperatorTok{$}\NormalTok{sdev}\OperatorTok{^}\DecValTok{2}\OperatorTok{/}\KeywordTok{sum}\NormalTok{(data}\OperatorTok{$}\NormalTok{sdev}\OperatorTok{^}\DecValTok{2}\NormalTok{)}\OperatorTok{*}\DecValTok{100}\NormalTok{,}\DecValTok{1}\NormalTok{)}
  \CommentTok{# main plot}
\NormalTok{  p1 <-}\StringTok{ }\KeywordTok{ggplot}\NormalTok{(dataDf,}\KeywordTok{aes}\NormalTok{(}\DataTypeTok{x=}\NormalTok{PC1, }\DataTypeTok{y=}\NormalTok{PC2)) }\OperatorTok{+}
\StringTok{    }\KeywordTok{theme_classic}\NormalTok{() }\OperatorTok{+}
\StringTok{    }\KeywordTok{geom_hline}\NormalTok{(}\DataTypeTok{yintercept =} \DecValTok{0}\NormalTok{, }\DataTypeTok{color =} \StringTok{"gray70"}\NormalTok{) }\OperatorTok{+}
\StringTok{    }\KeywordTok{geom_vline}\NormalTok{(}\DataTypeTok{xintercept =} \DecValTok{0}\NormalTok{, }\DataTypeTok{color =} \StringTok{"gray70"}\NormalTok{) }\OperatorTok{+}
\StringTok{    }\KeywordTok{geom_point}\NormalTok{(}\KeywordTok{aes}\NormalTok{(}\DataTypeTok{color =}\NormalTok{ Group), }\DataTypeTok{alpha =} \FloatTok{0.55}\NormalTok{, }\DataTypeTok{size =} \DecValTok{3}\NormalTok{) }\OperatorTok{+}
\StringTok{    }\KeywordTok{coord_cartesian}\NormalTok{(}\DataTypeTok{xlim =} \KeywordTok{c}\NormalTok{(}\KeywordTok{min}\NormalTok{(data}\OperatorTok{$}\NormalTok{x[,}\DecValTok{1}\NormalTok{])}\OperatorTok{-}\DecValTok{5}\NormalTok{,}\KeywordTok{max}\NormalTok{(data}\OperatorTok{$}\NormalTok{x[,}\DecValTok{1}\NormalTok{])}\OperatorTok{+}\DecValTok{5}\NormalTok{)) }\OperatorTok{+}
\StringTok{    }\KeywordTok{scale_fill_discrete}\NormalTok{(}\DataTypeTok{name =} \StringTok{"Group"}\NormalTok{)}
  \CommentTok{# avoiding labels superposition}
\NormalTok{  p1 }\OperatorTok{+}\StringTok{ }\KeywordTok{geom_text_repel}\NormalTok{(}\KeywordTok{aes}\NormalTok{(}\DataTypeTok{y =}\NormalTok{ PC2 }\OperatorTok{+}\StringTok{ }\FloatTok{0.25}\NormalTok{, }\DataTypeTok{label =}\NormalTok{ labels),}\DataTypeTok{segment.size =} \FloatTok{0.25}\NormalTok{, }\DataTypeTok{size =}\NormalTok{ size) }\OperatorTok{+}\StringTok{ }
\StringTok{    }\KeywordTok{labs}\NormalTok{(}\DataTypeTok{x =} \KeywordTok{c}\NormalTok{(}\KeywordTok{paste}\NormalTok{(}\StringTok{"PC1"}\NormalTok{,loads[}\DecValTok{1}\NormalTok{],}\StringTok{"%"}\NormalTok{)),}\DataTypeTok{y=}\KeywordTok{c}\NormalTok{(}\KeywordTok{paste}\NormalTok{(}\StringTok{"PC2"}\NormalTok{,loads[}\DecValTok{2}\NormalTok{],}\StringTok{"%"}\NormalTok{))) }\OperatorTok{+}\StringTok{  }
\StringTok{    }\KeywordTok{ggtitle}\NormalTok{(}\KeywordTok{paste}\NormalTok{(}\StringTok{"Principal Component Analysis for: "}\NormalTok{,title,}\DataTypeTok{sep=}\StringTok{" "}\NormalTok{))}\OperatorTok{+}\StringTok{ }
\StringTok{    }\KeywordTok{theme}\NormalTok{(}\DataTypeTok{plot.title =} \KeywordTok{element_text}\NormalTok{(}\DataTypeTok{hjust =} \FloatTok{0.5}\NormalTok{)) }\OperatorTok{+}
\StringTok{    }\KeywordTok{scale_color_manual}\NormalTok{(}\DataTypeTok{values=}\NormalTok{colores)}
\NormalTok{\}}

\CommentTok{##PCA raw data}
\KeywordTok{plotPCA3}\NormalTok{(}\KeywordTok{exprs}\NormalTok{(rawData), }\DataTypeTok{labels =}\NormalTok{ targets}\OperatorTok{$}\NormalTok{ShortName, }\DataTypeTok{factor =}\NormalTok{ targets}\OperatorTok{$}\NormalTok{Group, }
         \DataTypeTok{title=}\StringTok{"Raw data"}\NormalTok{, }\DataTypeTok{scale =} \OtherTok{FALSE}\NormalTok{, }\DataTypeTok{size =} \DecValTok{3}\NormalTok{, }
         \DataTypeTok{colores =} \KeywordTok{c}\NormalTok{(}\StringTok{"red"}\NormalTok{, }\StringTok{"blue"}\NormalTok{, }\StringTok{"green"}\NormalTok{, }\StringTok{"yellow"}\NormalTok{))}


\CommentTok{##Boxplot raw data}
\KeywordTok{boxplot}\NormalTok{(rawData, }\DataTypeTok{cex.axis=}\FloatTok{0.5}\NormalTok{, }\DataTypeTok{las=}\DecValTok{2}\NormalTok{,  }\DataTypeTok{which=}\StringTok{"all"}\NormalTok{, }
        \DataTypeTok{col =} \KeywordTok{c}\NormalTok{(}\KeywordTok{rep}\NormalTok{(}\StringTok{"red"}\NormalTok{, }\DecValTok{3}\NormalTok{), }\KeywordTok{rep}\NormalTok{(}\StringTok{"blue"}\NormalTok{, }\DecValTok{3}\NormalTok{), }\KeywordTok{rep}\NormalTok{(}\StringTok{"green"}\NormalTok{, }\DecValTok{3}\NormalTok{), }\KeywordTok{rep}\NormalTok{(}\StringTok{"yellow"}\NormalTok{, }\DecValTok{3}\NormalTok{)),}
        \DataTypeTok{main=}\StringTok{"Distribution of raw intensity values"}\NormalTok{)}
  \CommentTok{#Se observa que los boxplot presentan diferencias aprecialbes}
  \CommentTok{#se decide normalizar la base de datos}




\CommentTok{###Normalización###}
\NormalTok{eset_rma <-}\StringTok{ }\KeywordTok{rma}\NormalTok{(rawData)}

\CommentTok{##Análisis de calidad con datos normalizdos}
\KeywordTok{arrayQualityMetrics}\NormalTok{(eset_rma, }\DataTypeTok{outdir =} \KeywordTok{file.path}\NormalTok{(}\StringTok{"./results"}\NormalTok{, }\StringTok{"QCDir.Norm"}\NormalTok{), }\DataTypeTok{force=}\OtherTok{TRUE}\NormalTok{)}


\CommentTok{##PCA Normalizado}
\KeywordTok{plotPCA3}\NormalTok{(}\KeywordTok{exprs}\NormalTok{(eset_rma), }\DataTypeTok{labels =}\NormalTok{ targets}\OperatorTok{$}\NormalTok{ShortName, }\DataTypeTok{factor =}\NormalTok{ targets}\OperatorTok{$}\NormalTok{Group, }
         \DataTypeTok{title=}\StringTok{"Normalized data"}\NormalTok{, }\DataTypeTok{scale =} \OtherTok{FALSE}\NormalTok{, }\DataTypeTok{size =} \DecValTok{3}\NormalTok{, }
         \DataTypeTok{colores =} \KeywordTok{c}\NormalTok{(}\StringTok{"red"}\NormalTok{, }\StringTok{"blue"}\NormalTok{, }\StringTok{"green"}\NormalTok{, }\StringTok{"yellow"}\NormalTok{))}

\CommentTok{##Box plot Normalizado}
\KeywordTok{boxplot}\NormalTok{(eset_rma, }\DataTypeTok{cex.axis=}\FloatTok{0.5}\NormalTok{, }\DataTypeTok{las=}\DecValTok{2}\NormalTok{,  }\DataTypeTok{which=}\StringTok{"all"}\NormalTok{, }
        \DataTypeTok{col =} \KeywordTok{c}\NormalTok{(}\KeywordTok{rep}\NormalTok{(}\StringTok{"red"}\NormalTok{, }\DecValTok{3}\NormalTok{), }\KeywordTok{rep}\NormalTok{(}\StringTok{"blue"}\NormalTok{, }\DecValTok{3}\NormalTok{), }\KeywordTok{rep}\NormalTok{(}\StringTok{"green"}\NormalTok{, }\DecValTok{3}\NormalTok{), }\KeywordTok{rep}\NormalTok{(}\StringTok{"yellow"}\NormalTok{, }\DecValTok{3}\NormalTok{)),}
        \DataTypeTok{main=}\StringTok{"Boxplot for arrays intensity: Normalized Data"}\NormalTok{)}


\CommentTok{###Batch Detection###}
\KeywordTok{library}\NormalTok{(pvca)}
\KeywordTok{pData}\NormalTok{(eset_rma) <-}\StringTok{ }\NormalTok{targets}
\CommentTok{#selección del umbral}
\NormalTok{pct_threshold <-}\StringTok{ }\FloatTok{0.6}
\CommentTok{#selección de los factores a analizar}
\NormalTok{batch.factors <-}\StringTok{ }\KeywordTok{c}\NormalTok{(}\StringTok{"Genotype"}\NormalTok{, }\StringTok{"AD"}\NormalTok{)}
\CommentTok{#Análisis}
\NormalTok{pvcaObj <-}\StringTok{ }\KeywordTok{pvcaBatchAssess}\NormalTok{ (eset_rma, batch.factors, pct_threshold)}


\CommentTok{##Fuentes de variación}
\NormalTok{bp <-}\StringTok{ }\KeywordTok{barplot}\NormalTok{(pvcaObj}\OperatorTok{$}\NormalTok{dat, }\DataTypeTok{xlab =} \StringTok{"Effects"}\NormalTok{,}
              \DataTypeTok{ylab =} \StringTok{"Weighted average proportion variance"}\NormalTok{,}
              \DataTypeTok{ylim=} \KeywordTok{c}\NormalTok{(}\DecValTok{0}\NormalTok{,}\FloatTok{1.1}\NormalTok{),}\DataTypeTok{col =} \KeywordTok{c}\NormalTok{(}\StringTok{"mediumorchid"}\NormalTok{), }\DataTypeTok{las=}\DecValTok{2}\NormalTok{,}
              \DataTypeTok{main=}\StringTok{"PVCA estimation"}\NormalTok{)}
\KeywordTok{axis}\NormalTok{(}\DecValTok{1}\NormalTok{, }\DataTypeTok{at =}\NormalTok{ bp, }\DataTypeTok{labels =}\NormalTok{ pvcaObj}\OperatorTok{$}\NormalTok{label, }\DataTypeTok{cex.axis =} \FloatTok{0.55}\NormalTok{, }\DataTypeTok{las=}\DecValTok{2}\NormalTok{)}
\NormalTok{values =}\StringTok{ }\NormalTok{pvcaObj}\OperatorTok{$}\NormalTok{dat}
\NormalTok{new_values =}\StringTok{ }\KeywordTok{round}\NormalTok{(values , }\DecValTok{3}\NormalTok{)}
\KeywordTok{text}\NormalTok{(bp,pvcaObj}\OperatorTok{$}\NormalTok{dat,}\DataTypeTok{labels =}\NormalTok{ new_values, }\DataTypeTok{pos=}\DecValTok{3}\NormalTok{, }\DataTypeTok{cex =} \FloatTok{0.5}\NormalTok{)}


\CommentTok{##Niveles de la variación estándar}
\NormalTok{sds <-}\StringTok{ }\KeywordTok{apply}\NormalTok{ (}\KeywordTok{exprs}\NormalTok{(eset_rma), }\DecValTok{1}\NormalTok{, sd)}
\NormalTok{sdsO<-}\StringTok{ }\KeywordTok{sort}\NormalTok{(sds)}
\KeywordTok{plot}\NormalTok{(}\DecValTok{1}\OperatorTok{:}\KeywordTok{length}\NormalTok{(sdsO), sdsO, }\DataTypeTok{main=}\StringTok{"Distribution of variability for all genes"}\NormalTok{,}
     \DataTypeTok{sub=}\StringTok{"Vertical lines represent 90% and 95% percentiles"}\NormalTok{,}
     \DataTypeTok{xlab=}\StringTok{"Gene index (from least to most variable)"}\NormalTok{, }\DataTypeTok{ylab=}\StringTok{"Standard deviation"}\NormalTok{)}
\KeywordTok{abline}\NormalTok{(}\DataTypeTok{v=}\KeywordTok{length}\NormalTok{(sds)}\OperatorTok{*}\KeywordTok{c}\NormalTok{(}\FloatTok{0.9}\NormalTok{,}\FloatTok{0.95}\NormalTok{))}



\CommentTok{###Filtrado de datos###}
\KeywordTok{library}\NormalTok{(genefilter)}
\KeywordTok{library}\NormalTok{(mogene21sttranscriptcluster.db)}
\KeywordTok{annotation}\NormalTok{(eset_rma) <-}\StringTok{ "mogene21sttranscriptcluster.db"}
\NormalTok{filtered <-}\StringTok{ }\KeywordTok{nsFilter}\NormalTok{(eset_rma, }
                     \DataTypeTok{require.entrez =} \OtherTok{TRUE}\NormalTok{, }\DataTypeTok{remove.dupEntrez =} \OtherTok{TRUE}\NormalTok{,}
                     \DataTypeTok{var.filter=}\OtherTok{TRUE}\NormalTok{, }\DataTypeTok{var.func=}\NormalTok{IQR, }\DataTypeTok{var.cutoff=}\FloatTok{0.75}\NormalTok{, }
                     \DataTypeTok{filterByQuantile=}\OtherTok{TRUE}\NormalTok{, }\DataTypeTok{feature.exclude =} \StringTok{"^AFFX"}\NormalTok{)}


\CommentTok{##Resultados del filtrado}
\KeywordTok{print}\NormalTok{(filtered}\OperatorTok{$}\NormalTok{filter.log)}
\NormalTok{eset_filtered <-filtered}\OperatorTok{$}\NormalTok{eset}

\CommentTok{##Exportando los datos}
\KeywordTok{write.csv}\NormalTok{(}\KeywordTok{exprs}\NormalTok{(eset_rma), }\DataTypeTok{file=}\StringTok{"./results/normalized.Data.csv"}\NormalTok{)}
\KeywordTok{write.csv}\NormalTok{(}\KeywordTok{exprs}\NormalTok{(eset_filtered), }\DataTypeTok{file=}\StringTok{"./results/normalized.Filtered.Data.csv"}\NormalTok{)}
\KeywordTok{save}\NormalTok{(eset_rma, eset_filtered, }\DataTypeTok{file=}\StringTok{"./results/normalized.Data.Rda"}\NormalTok{)}


\CommentTok{###Identificación de genes diferencialmente expresados###}
\CommentTok{##Diseño de la matriz}
\KeywordTok{library}\NormalTok{(limma)}
\NormalTok{designMat<-}\StringTok{ }\KeywordTok{model.matrix}\NormalTok{(}\OperatorTok{~}\DecValTok{0}\OperatorTok{+}\NormalTok{Group, }\KeywordTok{pData}\NormalTok{(eset_filtered))}
\KeywordTok{colnames}\NormalTok{(designMat) <-}\StringTok{ }\KeywordTok{c}\NormalTok{(}\StringTok{"A3xTg.KO"}\NormalTok{, }\StringTok{"A3xTg.WT"}\NormalTok{, }\StringTok{"APPNL.KO"}\NormalTok{, }\StringTok{"APPNL.WT"}\NormalTok{)}
\KeywordTok{print}\NormalTok{(designMat)}

\CommentTok{##Matriz de comparación}
\NormalTok{cont.matrix <-}\StringTok{ }\KeywordTok{makeContrasts}\NormalTok{ (}\DataTypeTok{KOvsWT.3xTg =}\NormalTok{ A3xTg.KO}\OperatorTok{-}\NormalTok{A3xTg.WT,}
                              \DataTypeTok{KOvsWT.APP =}\NormalTok{ APPNL.KO}\OperatorTok{-}\NormalTok{APPNL.WT,}
                              \DataTypeTok{INT =}\NormalTok{ (A3xTg.KO}\OperatorTok{-}\NormalTok{A3xTg.WT) }\OperatorTok{-}\StringTok{ }\NormalTok{(APPNL.KO}\OperatorTok{-}\NormalTok{APPNL.WT),}
                              \DataTypeTok{levels=}\NormalTok{designMat)}
\KeywordTok{print}\NormalTok{(cont.matrix)}

\CommentTok{##Modelo lineal}
\KeywordTok{library}\NormalTok{(limma)}
\NormalTok{fit<-}\KeywordTok{lmFit}\NormalTok{(eset_filtered, designMat)}
\NormalTok{fit.main<-}\KeywordTok{contrasts.fit}\NormalTok{(fit, cont.matrix)}
\NormalTok{fit.main<-}\KeywordTok{eBayes}\NormalTok{(fit.main)}
\KeywordTok{class}\NormalTok{(fit.main)}

\CommentTok{##Top Tab A3xTg.KO-A3xTg.WT - Efecto Especie 3xTg}
\NormalTok{topTab_KOvsWT}\FloatTok{.3}\NormalTok{xTg <-}\StringTok{ }\KeywordTok{topTable}\NormalTok{ (fit.main, }\DataTypeTok{number=}\KeywordTok{nrow}\NormalTok{(fit.main), }\DataTypeTok{coef=}\StringTok{"KOvsWT.3xTg"}\NormalTok{, }\DataTypeTok{adjust=}\StringTok{"fdr"}\NormalTok{) }
\KeywordTok{head}\NormalTok{(topTab_KOvsWT}\FloatTok{.3}\NormalTok{xTg)}

\CommentTok{##Top Tab APPNL.KO-APPNL.WT - Efecto Especie APPNL}
\NormalTok{topTab_KOvsWT.APP <-}\StringTok{ }\KeywordTok{topTable}\NormalTok{ (fit.main, }\DataTypeTok{number=}\KeywordTok{nrow}\NormalTok{(fit.main), }\DataTypeTok{coef=}\StringTok{"KOvsWT.APP"}\NormalTok{, }\DataTypeTok{adjust=}\StringTok{"fdr"}\NormalTok{) }
\KeywordTok{head}\NormalTok{(topTab_KOvsWT.APP)}

\CommentTok{##Top Tab (A3xTg.KO-A3xTg.WT) - (APPNL.KO-APPNL.WT) - Efecto interacción}
\NormalTok{topTab_INT  <-}\StringTok{ }\KeywordTok{topTable}\NormalTok{ (fit.main, }\DataTypeTok{number=}\KeywordTok{nrow}\NormalTok{(fit.main), }\DataTypeTok{coef=}\StringTok{"INT"}\NormalTok{, }\DataTypeTok{adjust=}\StringTok{"fdr"}\NormalTok{) }
\KeywordTok{head}\NormalTok{(topTab_INT)}

\CommentTok{###Anotación de los resultados###}

\CommentTok{##Generando la función de anotación}
\NormalTok{annotatedTopTable <-}\StringTok{ }\ControlFlowTok{function}\NormalTok{(topTab, anotPackage)}
\NormalTok{\{}
\NormalTok{  topTab <-}\StringTok{ }\KeywordTok{cbind}\NormalTok{(}\DataTypeTok{PROBEID=}\KeywordTok{rownames}\NormalTok{(topTab), topTab)}
\NormalTok{  myProbes <-}\StringTok{ }\KeywordTok{rownames}\NormalTok{(topTab)}
\NormalTok{  thePackage <-}\StringTok{ }\KeywordTok{eval}\NormalTok{(}\KeywordTok{parse}\NormalTok{(}\DataTypeTok{text =}\NormalTok{ anotPackage))}
\NormalTok{  geneAnots <-}\StringTok{ }\KeywordTok{select}\NormalTok{(thePackage, myProbes, }\KeywordTok{c}\NormalTok{(}\StringTok{"SYMBOL"}\NormalTok{, }\StringTok{"ENTREZID"}\NormalTok{, }\StringTok{"GENENAME"}\NormalTok{))}
\NormalTok{  annotatedTopTab<-}\StringTok{ }\KeywordTok{merge}\NormalTok{(}\DataTypeTok{x=}\NormalTok{geneAnots, }\DataTypeTok{y=}\NormalTok{topTab, }\DataTypeTok{by.x=}\StringTok{"PROBEID"}\NormalTok{, }\DataTypeTok{by.y=}\StringTok{"PROBEID"}\NormalTok{)}
  \KeywordTok{return}\NormalTok{(annotatedTopTab)}
\NormalTok{\}}

\CommentTok{##Anotando las tablas}
\NormalTok{topAnnotated_KOvsWT}\FloatTok{.3}\NormalTok{xTg <-}\StringTok{ }\KeywordTok{annotatedTopTable}\NormalTok{(topTab_KOvsWT}\FloatTok{.3}\NormalTok{xTg,}
                                              \DataTypeTok{anotPackage=}\StringTok{"mogene21sttranscriptcluster.db"}\NormalTok{)}
\NormalTok{topAnnotated_KOvsWT.APP <-}\StringTok{ }\KeywordTok{annotatedTopTable}\NormalTok{(topTab_KOvsWT.APP,}
                                              \DataTypeTok{anotPackage=}\StringTok{"mogene21sttranscriptcluster.db"}\NormalTok{)}
\NormalTok{topAnnotated_INT <-}\StringTok{ }\KeywordTok{annotatedTopTable}\NormalTok{(topTab_INT,}
                                      \DataTypeTok{anotPackage=}\StringTok{"mogene21sttranscriptcluster.db"}\NormalTok{)}

\KeywordTok{write.csv}\NormalTok{(topAnnotated_KOvsWT}\FloatTok{.3}\NormalTok{xTg, }\DataTypeTok{file=}\StringTok{"./results/topAnnotated_KOvsWT.3xTg.csv"}\NormalTok{)}
\KeywordTok{write.csv}\NormalTok{(topAnnotated_KOvsWT.APP, }\DataTypeTok{file=}\StringTok{"./results/topAnnotated_KOvsWT.APP.csv"}\NormalTok{)}
\KeywordTok{write.csv}\NormalTok{(topAnnotated_INT, }\DataTypeTok{file=}\StringTok{"./results/topAnnotated_INT.csv"}\NormalTok{)}


\CommentTok{##Muestra de confirmación}
\KeywordTok{head}\NormalTok{(topAnnotated_INT[}\DecValTok{1}\OperatorTok{:}\DecValTok{5}\NormalTok{,}\DecValTok{1}\OperatorTok{:}\DecValTok{4}\NormalTok{])}
  \CommentTok{#Se confirma que la anotación ocurrió}

\CommentTok{##Gráfico de volcán}
\KeywordTok{library}\NormalTok{(mogene21sttranscriptcluster.db)}
\NormalTok{geneSymbols <-}\StringTok{ }\KeywordTok{select}\NormalTok{(mogene21sttranscriptcluster.db, }\KeywordTok{rownames}\NormalTok{(fit.main), }\KeywordTok{c}\NormalTok{(}\StringTok{"SYMBOL"}\NormalTok{))}
\NormalTok{SYMBOLS<-}\StringTok{ }\NormalTok{geneSymbols}\OperatorTok{$}\NormalTok{SYMBOL}
\KeywordTok{volcanoplot}\NormalTok{(fit.main, }\DataTypeTok{coef=}\DecValTok{1}\NormalTok{, }\DataTypeTok{highlight=}\DecValTok{8}\NormalTok{, }\DataTypeTok{names=}\NormalTok{SYMBOLS, }
            \DataTypeTok{main=}\KeywordTok{paste}\NormalTok{(}\StringTok{"Differentially expressed genes"}\NormalTok{, }\KeywordTok{colnames}\NormalTok{(cont.matrix)[}\DecValTok{1}\NormalTok{], }\DataTypeTok{sep=}\StringTok{"}\CharTok{\textbackslash{}n}\StringTok{"}\NormalTok{))}
\KeywordTok{abline}\NormalTok{(}\DataTypeTok{v=}\KeywordTok{c}\NormalTok{(}\OperatorTok{-}\DecValTok{1}\NormalTok{,}\DecValTok{1}\NormalTok{))}

  \CommentTok{#Revisar función de los genes resaltados}

\CommentTok{##Comparación múltiple}
\KeywordTok{library}\NormalTok{(limma)}
\NormalTok{res<-}\KeywordTok{decideTests}\NormalTok{(fit.main, }\DataTypeTok{method=}\StringTok{"separate"}\NormalTok{, }\DataTypeTok{adjust.method=}\StringTok{"fdr"}\NormalTok{, }\DataTypeTok{p.value=}\FloatTok{0.1}\NormalTok{, }\DataTypeTok{lfc=}\DecValTok{1}\NormalTok{)}

\CommentTok{##Resumen de la comparación múltiple}
\NormalTok{sum.res.rows<-}\KeywordTok{apply}\NormalTok{(}\KeywordTok{abs}\NormalTok{(res),}\DecValTok{1}\NormalTok{,sum)}
\NormalTok{res.selected<-res[sum.res.rows}\OperatorTok{!=}\DecValTok{0}\NormalTok{,] }
\KeywordTok{print}\NormalTok{(}\KeywordTok{summary}\NormalTok{(res))}


\CommentTok{##Diagrama Venn}
\KeywordTok{vennDiagram}\NormalTok{ (res.selected[,}\DecValTok{1}\OperatorTok{:}\DecValTok{3}\NormalTok{], }\DataTypeTok{cex=}\FloatTok{0.9}\NormalTok{)}
\KeywordTok{title}\NormalTok{(}\StringTok{"Genes in common between the three comparisons}\CharTok{\textbackslash{}n}\StringTok{ Genes selected with FDR < 0.1 and logFC > 1"}\NormalTok{)}


\CommentTok{##Datos para el mapa de calor}
\NormalTok{probesInHeatmap <-}\StringTok{ }\KeywordTok{rownames}\NormalTok{(res.selected)}
\NormalTok{HMdata <-}\StringTok{ }\KeywordTok{exprs}\NormalTok{(eset_filtered)[}\KeywordTok{rownames}\NormalTok{(}\KeywordTok{exprs}\NormalTok{(eset_filtered)) }\OperatorTok\StringTok{ }\NormalTok{probesInHeatmap,]}

\NormalTok{geneSymbols <-}\StringTok{ }\KeywordTok{select}\NormalTok{(mogene21sttranscriptcluster.db, }\KeywordTok{rownames}\NormalTok{(HMdata), }\KeywordTok{c}\NormalTok{(}\StringTok{"SYMBOL"}\NormalTok{))}
\NormalTok{SYMBOLS<-}\StringTok{ }\NormalTok{geneSymbols}\OperatorTok{$}\NormalTok{SYMBOL}
\KeywordTok{rownames}\NormalTok{(HMdata) <-}\StringTok{ }\NormalTok{SYMBOLS}
\KeywordTok{write.csv}\NormalTok{(HMdata, }\DataTypeTok{file =} \KeywordTok{file.path}\NormalTok{(}\StringTok{"./results/data4Heatmap.csv"}\NormalTok{))}

\NormalTok{my_palette <-}\StringTok{ }\KeywordTok{colorRampPalette}\NormalTok{(}\KeywordTok{c}\NormalTok{(}\StringTok{"black"}\NormalTok{, }\StringTok{"white"}\NormalTok{))(}\DataTypeTok{n =} \DecValTok{299}\NormalTok{)}

\CommentTok{##Mapa de Calor}
\KeywordTok{library}\NormalTok{(gplots)}
\KeywordTok{heatmap.2}\NormalTok{(HMdata,}
          \DataTypeTok{Rowv =} \OtherTok{FALSE}\NormalTok{,}
          \DataTypeTok{Colv =} \OtherTok{FALSE}\NormalTok{,}
          \DataTypeTok{main =} \StringTok{"Differentially expressed genes }\CharTok{\textbackslash{}n}\StringTok{ FDR < 0,1, logFC >=1"}\NormalTok{,}
          \DataTypeTok{scale =} \StringTok{"row"}\NormalTok{,}
          \DataTypeTok{col =}\NormalTok{ my_palette,}
          \DataTypeTok{sepcolor =} \StringTok{"white"}\NormalTok{,}
          \DataTypeTok{sepwidth =} \KeywordTok{c}\NormalTok{(}\FloatTok{0.05}\NormalTok{,}\FloatTok{0.05}\NormalTok{),}
          \DataTypeTok{cexRow =} \FloatTok{0.5}\NormalTok{,}
          \DataTypeTok{cexCol =} \FloatTok{0.9}\NormalTok{,}
          \DataTypeTok{key =} \OtherTok{TRUE}\NormalTok{,}
          \DataTypeTok{keysize =} \FloatTok{1.5}\NormalTok{,}
          \DataTypeTok{density.info =} \StringTok{"histogram"}\NormalTok{,}
          \DataTypeTok{ColSideColors =} \KeywordTok{c}\NormalTok{(}\KeywordTok{rep}\NormalTok{(}\StringTok{"red"}\NormalTok{,}\DecValTok{3}\NormalTok{),}\KeywordTok{rep}\NormalTok{(}\StringTok{"blue"}\NormalTok{,}\DecValTok{3}\NormalTok{), }\KeywordTok{rep}\NormalTok{(}\StringTok{"green"}\NormalTok{,}\DecValTok{3}\NormalTok{), }\KeywordTok{rep}\NormalTok{(}\StringTok{"yellow"}\NormalTok{,}\DecValTok{3}\NormalTok{)),}
          \DataTypeTok{tracecol =} \OtherTok{NULL}\NormalTok{,}
          \DataTypeTok{dendrogram =} \StringTok{"none"}\NormalTok{,}
          \DataTypeTok{srtCol =} \DecValTok{30}\NormalTok{)}

\CommentTok{##Mapa de calor jerarquizado}
\KeywordTok{heatmap.2}\NormalTok{(HMdata,}
          \DataTypeTok{Rowv =} \OtherTok{TRUE}\NormalTok{,}
          \DataTypeTok{Colv =} \OtherTok{TRUE}\NormalTok{,}
          \DataTypeTok{dendrogram =} \StringTok{"both"}\NormalTok{,}
          \DataTypeTok{main =} \StringTok{"Differentially expressed genes }\CharTok{\textbackslash{}n}\StringTok{ FDR < 0,1, logFC >=1"}\NormalTok{,}
          \DataTypeTok{scale =} \StringTok{"row"}\NormalTok{,}
          \DataTypeTok{col =}\NormalTok{ my_palette,}
          \DataTypeTok{sepcolor =} \StringTok{"white"}\NormalTok{,}
          \DataTypeTok{sepwidth =} \KeywordTok{c}\NormalTok{(}\FloatTok{0.05}\NormalTok{,}\FloatTok{0.05}\NormalTok{),}
          \DataTypeTok{cexRow =} \FloatTok{0.5}\NormalTok{,}
          \DataTypeTok{cexCol =} \FloatTok{0.9}\NormalTok{,}
          \DataTypeTok{key =} \OtherTok{TRUE}\NormalTok{,}
          \DataTypeTok{keysize =} \FloatTok{1.5}\NormalTok{,}
          \DataTypeTok{density.info =} \StringTok{"histogram"}\NormalTok{,}
          \DataTypeTok{ColSideColors =} \KeywordTok{c}\NormalTok{(}\KeywordTok{rep}\NormalTok{(}\StringTok{"red"}\NormalTok{,}\DecValTok{3}\NormalTok{),}\KeywordTok{rep}\NormalTok{(}\StringTok{"blue"}\NormalTok{,}\DecValTok{3}\NormalTok{), }\KeywordTok{rep}\NormalTok{(}\StringTok{"green"}\NormalTok{,}\DecValTok{3}\NormalTok{), }\KeywordTok{rep}\NormalTok{(}\StringTok{"yellow"}\NormalTok{,}\DecValTok{3}\NormalTok{)),}
          \DataTypeTok{tracecol =} \OtherTok{NULL}\NormalTok{,}
          \DataTypeTok{srtCol =} \DecValTok{30}\NormalTok{)}


\CommentTok{###Significancia biológica###}
\CommentTok{#Preparación de datos}

\NormalTok{listOfTables <-}\StringTok{ }\KeywordTok{list}\NormalTok{(}\DataTypeTok{KOvsWT.3xTg =}\NormalTok{ topTab_KOvsWT}\FloatTok{.3}\NormalTok{xTg, }
                     \DataTypeTok{KOvsWT.APP  =}\NormalTok{ topTab_KOvsWT.APP, }
                     \DataTypeTok{INT =}\NormalTok{ topTab_INT)}
\NormalTok{listOfSelected <-}\StringTok{ }\KeywordTok{list}\NormalTok{()}

\CommentTok{#Error en esta parte}

\ControlFlowTok{for}\NormalTok{ (i }\ControlFlowTok{in} \DecValTok{1}\OperatorTok{:}\KeywordTok{length}\NormalTok{(listOfTables))\{}
  \CommentTok{# select the toptable}
\NormalTok{  topTab <-}\StringTok{ }\NormalTok{listOfTables[[i]]}
  \CommentTok{# select the genes to be included in the analysis}
\NormalTok{  whichGenes<-topTab[}\StringTok{"adj.P.Val"}\NormalTok{]}\OperatorTok{<}\FloatTok{0.15}
  \CommentTok{#selectedIDs <- (topTab)[whichGenes,1]}
\NormalTok{  selectedIDs <-}\KeywordTok{rownames}\NormalTok{(topTab)[whichGenes]}
  \CommentTok{# convert the ID to Entrez}
\NormalTok{  EntrezIDs<-}\StringTok{ }\KeywordTok{select}\NormalTok{(mogene21sttranscriptcluster.db, selectedIDs, }\KeywordTok{c}\NormalTok{(}\StringTok{"ENTREZID"}\NormalTok{))}
\NormalTok{  EntrezIDs <-}\StringTok{ }\NormalTok{EntrezIDs}\OperatorTok{$}\NormalTok{ENTREZID}
\NormalTok{  listOfSelected[[i]] <-}\StringTok{ }\NormalTok{EntrezIDs}
  \KeywordTok{names}\NormalTok{(listOfSelected)[i] <-}\StringTok{ }\KeywordTok{names}\NormalTok{(listOfTables)[i]}
\NormalTok{\}}
\KeywordTok{sapply}\NormalTok{(listOfSelected, length)}


\CommentTok{## --------------------------------------------------------------------------------------------------}
\NormalTok{mapped_genes2GO <-}\StringTok{ }\KeywordTok{mappedkeys}\NormalTok{(org.Mm.egGO)}
\NormalTok{mapped_genes2KEGG <-}\StringTok{ }\KeywordTok{mappedkeys}\NormalTok{(org.Mm.egPATH)}
\NormalTok{mapped_genes <-}\StringTok{ }\KeywordTok{union}\NormalTok{(mapped_genes2GO , mapped_genes2KEGG)}


\CommentTok{##Señal Biológica}
\KeywordTok{library}\NormalTok{(ReactomePA)}

\NormalTok{listOfData <-}\StringTok{ }\NormalTok{listOfSelected[}\DecValTok{1}\OperatorTok{:}\DecValTok{2}\NormalTok{]}
\NormalTok{comparisonsNames <-}\StringTok{ }\KeywordTok{names}\NormalTok{(listOfData)}
\NormalTok{universe <-}\StringTok{ }\NormalTok{mapped_genes}

\ControlFlowTok{for}\NormalTok{ (i }\ControlFlowTok{in} \DecValTok{1}\OperatorTok{:}\KeywordTok{length}\NormalTok{(listOfData))\{}
\NormalTok{  genesIn <-}\StringTok{ }\NormalTok{listOfData[[i]]}
\NormalTok{  comparison <-}\StringTok{ }\NormalTok{comparisonsNames[i]}
\NormalTok{  enrich.result <-}\StringTok{ }\KeywordTok{enrichPathway}\NormalTok{(}\DataTypeTok{gene =}\NormalTok{ genesIn,}
                                 \DataTypeTok{pvalueCutoff =} \FloatTok{0.05}\NormalTok{,}
                                 \DataTypeTok{readable =}\NormalTok{ T,}
                                 \DataTypeTok{pAdjustMethod =} \StringTok{"BH"}\NormalTok{,}
                                 \DataTypeTok{organism =} \StringTok{"mouse"}\NormalTok{,}
                                 \DataTypeTok{universe =}\NormalTok{ universe)}
  
  \KeywordTok{cat}\NormalTok{(}\StringTok{"##########################"}\NormalTok{)}
  \KeywordTok{cat}\NormalTok{(}\StringTok{"}\CharTok{\textbackslash{}n}\StringTok{Comparison: "}\NormalTok{, comparison,}\StringTok{"}\CharTok{\textbackslash{}n}\StringTok{"}\NormalTok{)}
  \KeywordTok{print}\NormalTok{(}\KeywordTok{head}\NormalTok{(enrich.result))}
  
  \ControlFlowTok{if}\NormalTok{ (}\KeywordTok{length}\NormalTok{(}\KeywordTok{rownames}\NormalTok{(enrich.result}\OperatorTok{@}\NormalTok{result)) }\OperatorTok{!=}\StringTok{ }\DecValTok{0}\NormalTok{) \{}
    \KeywordTok{write.csv}\NormalTok{(}\KeywordTok{as.data.frame}\NormalTok{(enrich.result), }
              \DataTypeTok{file =}\KeywordTok{paste0}\NormalTok{(}\StringTok{"./results/"}\NormalTok{,}\StringTok{"ReactomePA.Results."}\NormalTok{,comparison,}\StringTok{".csv"}\NormalTok{), }
              \DataTypeTok{row.names =} \OtherTok{FALSE}\NormalTok{)}
    
    \KeywordTok{print}\NormalTok{(}\KeywordTok{barplot}\NormalTok{(enrich.result, }\DataTypeTok{showCategory =} \DecValTok{15}\NormalTok{, }\DataTypeTok{font.size =} \DecValTok{4}\NormalTok{, }
                  \DataTypeTok{title =} \KeywordTok{paste0}\NormalTok{(}\StringTok{"Reactome Pathway Analysis for "}\NormalTok{, comparison,}\StringTok{". Barplot"}\NormalTok{)))}
    
    \KeywordTok{print}\NormalTok{(}\KeywordTok{cnetplot}\NormalTok{(enrich.result, }\DataTypeTok{categorySize =} \StringTok{"geneNum"}\NormalTok{, }\DataTypeTok{schowCategory =} \DecValTok{15}\NormalTok{, }
                   \DataTypeTok{vertex.label.cex =} \FloatTok{0.75}\NormalTok{))}
    
\NormalTok{  \}}
\NormalTok{\}}

\CommentTok{## Network obtained from the Reactome enrichment analysis on the list obtained from the comparison between KO and WT in RT}
\KeywordTok{cnetplot}\NormalTok{(enrich.result, }\DataTypeTok{categorySize =} \StringTok{"geneNum"}\NormalTok{, }\DataTypeTok{schowCategory =} \DecValTok{15}\NormalTok{, }
         \DataTypeTok{vertex.label.cex =} \FloatTok{0.75}\NormalTok{)}
\end{Highlighting}
\end{Shaded}

\end{document}
